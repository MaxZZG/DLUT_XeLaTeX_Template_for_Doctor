% !TEX TS-program = xelatex
% !TEX encoding = UTF-8 Unicode

%%%%%%%%%%%%%%%%%%%%%%%%%%%%%%%%%%%%%%%%%%%%%%%%%%%%%%%%%%%%%%%%%%%%%%
%
%	大连理工大学博士论文 XeLaTeX 模版 —— 封面文件 cover.tex
%	版本:0.71
%	最后更新:2010.12.22
%	修改者:Yuri (E-mail: yuri_1985@163.com)
%	编译环境:Ubuntu 10.04 + TeXLive 2010 + TeXworks
%             Windows XP SP3 + CTeXLive 2009 + WinEdt 5.6
%
%%%%%%%%%%%%%%%%%%%%%%%%%%%%%%%%%%%%%%%%%%%%%%%%%%%%%%%%%%%%%%%%%%%%%%

\cdegree{博~~士~~学~~位~~论~~文}
\ctitle{大连理工大学博士学位论文~\XeLaTeX{}~模版}
\etitle{The \XeLaTeX{} Template of Doctor Degree Thesis  of DUT}

% 根据需要添加字符间距
\csubject{{\quad\;}计~算~数~学~专~业}
\cauthor{{\quad\;}张~~ ~~三}
\cauthorno{{\quad\;}20801000}
\csupervisor{{\quad\;}王~~老~~五 副教授}

% 这里默认使用最后编译的时间,也可自行给定日期,注意汉字和数字之间的空格。
\cdate{{\quad\;}\the\year~年~\the\month~月~\the\day~日}

\cabstract{
本模版是根据大连理工大学博士学位论文格式规范制作的~\LaTeX~博士学位论文模板。

本模板是基于北京大学、清华大学、哈尔滨工业大学等高校的硕博士论文模板,
并按照大连理工大学博士学位论文格式规范开发的~\LaTeX~论文模板,
经过完善和修改,目前已经基本满足了论文规范的要求,
而且易用性良好,功能强大。不过,可能还存在着一些问题,
欢迎大家积极使用本模版,反馈遇到的问题,以便不断对其进行改进。

当然这个模板仅仅是一个开始,希望有更多的人能够参与进来,
不断改进准确性、易用性和较好的可维护性,造福需要的兄弟姐妹们。
总体上来说,当前这个模板还是很值得推荐使用的。

本模板的目的旨在推广~\LaTeX~这一优秀的排版软件在大工(尤其是数学相关专业)的应用,
为广大同学提供一个方便、美观的论文模板,减少论文撰写格式方面的麻烦。

以下顺便补充一些研究生院所提供的~Word~模版中的注意事项
(略去已经嵌入到此模版中的内容):

\begin{asparaenum}
    \item 论文摘要是学位论文的缩影,文字要简练、明确。内容要包括目的、方法、结果和结论。
        单位制一律换算成国际标准计量单位制,除特别情况外,数字一律用阿拉伯数码。
        文中不允许出现插图。重要的表格可以写入;
    \item 篇幅以一页为限,字数为~600-800~字
        (工程硕士、MBA、EMBA、MPA~等专业学位论文字数为~400-500~字);
    \item 摘要正文后,列出~3-5~个关键词。
        关键词请尽量用《汉语主题词表》等词表提供的规范词。
        关键词词间用分号间隔,末尾不加标点,3-5~个。
\end{asparaenum}
}

\ckeywords{写作规范;排版格式;博士学位论文;\XeLaTeX{}模版}

\eabstract{
This is a \LaTeX{} template of doctor degree thesis of Dalian University of Technology,
which is built according to the required format.

内容应与“中文摘要”对应。使用第三人称,最好采用现在时态编写。
}

\ekeywords{Write Criterion; Typeset Format; Doctor's Degree Paper; \XeLaTeX{} Template}

\makecover 